\documentclass[a4paper,10pt]{article}
\usepackage[boxruled,vlined,portugues,figure,linesnumbered,longend]{algorithm2e}
\input{figures/transfig}
\usepackage[]{graphicx}
\usepackage{QED}
\usepackage{amssymb}


\newcommand*{\Assign}{\ensuremath\longleftarrow}
\newcommand{\putstring}[1]{\textsl{#1}}
\newtheorem{definition}{Definition}
\newtheorem{lemma}{Lemma}


\title{Dotted Suffix Trees\\A Structure for Approximate Text Indexing}
\author{L. P. Coelho \and A. Oliveira}

\begin{document}

\maketitle

\begin{abstract}
The problem we are addressing is the one of text indexing for approximate matching. We consider that we are given a text $\mathcal{T}$ which undergoes some preprocessing to generate an index. We can later query this index to identify the places where a string occurs up to a certain number of allowed substitutions $k$. We present a structure for this indexing which occupies space $O(n\log^kn)$ in the average case, indepedent of alphabet size, $n$ being the text size. This structure support searching in $O(m^{k+1})$ time, for patterns of size $m$, again independently of alphabet size. The construction of the structure has time bounded by $O(N|\Sigma|)$, where $N$ is the number of nodes in the index and $|\Sigma|$ the alphabet size.

\end{abstract}

\section{Introduction}
Since their introduction~\cite{weiner}, suffix trees have been one of the methods of choice for text indexing. Suffix trees solve the following problem: given a text which has been preprocessed, and a pattern, efficiently find the places in the text where this pattern occurs.  However, for many real-life problems this is too restrictive. In this class of problems, one is interested in finding places in the text where an approximate form of the pattern occurs. Several algorithms have been proposed to solve this problem. \cite{navarro:guided} presents a survey of the problem and existing approaches to solving it at the time that work was written. Recently there have been some further work in the field, notably~\cite{maass:text} which also contains an updated overview of methods for solving the problem in addition to presenting one itself.

In this work we present an approach which solves this problem based on an extention of suffix trees. The main advantage of this approach is that both the search and the index size are \emph{alphabet independent}, although the indexing time is not.

\subsection{Related Work}

Our work is related to efforts to index the text for approximate matching such as~\cite{maass:text,mosche}. The structure presented is superficially very similiar to the one presented in~\cite{inexact} as an \textit{inexact suffix tree} though that work is concerned with solving a different set of problems.

\subsection{Structure of the Paper}

In section~\ref{sec:structure} we present the structure and the way to use it for searching. In section~\ref{sec:build-1} we show the way to build a structure to handle one error only and analize this version of the algorithm. Then, we show, in section~\ref{sec:build-k} how to generalize this procedure for any number of errors and how the previous analisys also applies to any number of errors.



\section{The Indexing Structure}
\subsection{Intuition}

\begin{figure}
\includegraphics{figures/mississippi-0}
\caption{Suffix Tree for the string \putstring{mississippi}}%
\label{fig:mississippi-0}
\end{figure}

Suffix Trees have become a well known structure since their introduction in 1973~\cite{DBLP:conf/focs/Weiner73}. In Figure~\ref{fig:mississippi-0} we show the suffix tree for the string \putstring{mississippi}.

To extend the suffix tree, we consider that a node in the suffix tree corresponds to several points in the string at the same time (those points where the string on the path to the node occurs). We now add to each node an extra edge which corresponds to moving one character forward in all these points.

\subsection{Formal definition}

We assume that the reader is familiar with the basic concepts of suffix trees and strings and present the definitions below mainly to introduce the particular notation and terminology used in this paper.

\begin{definition}[Character, string]
Given a set $\Sigma$, we say that $s$ is a \emph{string over $\Sigma$} if $s$ is a possibly empty sequence of elements of $\Sigma$. The elements of the set $\Sigma$ are also called \emph{character}. The length of the string, denoted by $|s|$ is the number of elements it contains. We shall write $s_i$ for the $i$th element of the string. For the empty string, we write $\epsilon$.

The set of all string is denoted by $\Sigma^*$ and $\Sigma^+=\Sigma^*-\{\epsilon\}$.
\end{definition}

For denoting character we shall use letters from the beginning of the roman alphabet ($a$, $b$, $c$,\ldots) and for strings, we shall use letters from the end of the alphabet ($w$, $z$, \ldots).

\begin{definition}
$wx$ will denote the string formed by all the character of $w$ followed by all the characters of $x$ (the usual concatenation operation). $aw$ ($wa$) shall the denote the strings by the character $a$ before (after) the characters of $w$.

If $s = wxy$, then $w$ is a \emph{prefix} of $s$, $x$ is a \emph{substring} of $s$ (at position $s$) and $y$ is a \emph{suffix} of $s$ (at position $p$).
\end{definition}

\begin{definition}[Patricia tree]
$T$ is a Patricia tree if $T$ is a rooted tree with edge labels from $\Sigma^+$. For each $a \in \Sigma$ and every node $n$ in $T$, there is at most an edge leaving $n$ whose label is $aw$.

Each node in a Patricia tree has a path leading to it which forms a string. If the node $n$ has the leading path $w$, we shall also refer to $n$ as $\overline{w}$.
\end{definition}

The name Patricia is sometimes spelled PATRICIA, since it was formed from the initial of Practical Algorithm to Retrieve Information Coded in Alphanumeric~\cite{DBLP:journals/jacm/Morrison68}.

In what follows we will assume that there are two symbols (\$ and $.$) which are not part of $\Sigma$.

In this paper, we will be interested in substitution distance (aka Hamming distance):

\begin{definition}[Aproximate Match]
We say that the string $s$ matches the string $t$ at position $p$ with $k$ errors if we can make $k$ substitutions in $s$ such that $s$ is a substring of $t$ at position $p$.
\end{definition}


\begin{definition}[Pattern]
A pattern is a string over the extended alphabet $\Sigma\cup\{.\}$. We say that a pattern $s$ matches a string $t$ at position $p$, if for each dot in the pattern we can replace it by a character in the original alphabet to obtain $s'$ which is a substring of $t$ at position $p$.
\end{definition}

\begin{definition}[Suffix Tree]
A \emph{suffix tree} for a string $S$ is a $\Sigma^+$ whose leaf nodes (those without children) have paths corresponding to suffixes of the string $S\$ $.
\end{definition}

\begin{definition}[Suffix Link]
A suffix link in a suffix tree is a link from the node $\overline{aw}$ to the node $\overline{w}$. This node has the label $a$.
\end{definition}

In a suffix tree, it is possible to define a suffix link for each internal node. Both Ukkonen's~\cite{ukkonen92constructing} and McCreight's~\cite{mccreight} algorithms are linear time algorithms for constructing a suffix tree which includes suffix links. The suffix links are shown in Figure~\ref{fig:mississippi-0} as dotted links.

\begin{definition}[Occurence Set]
Given a node $\overline{w}$ in a suffix tree, we call it's \emph{occurrence set} the set of indexes in the original string where the string $w$ occurs.
\end{definition}

\begin{definition}[Position Set]
Given a node $\overline{w}$ in a suffix tree, it's \emph{position set} is the set formed by taking it's occurrence set and adding the length of $w$ to each element.
\end{definition}

For example, in the node $\overline{issi}$ in the suffix tree of Figure~\ref{fig:mississippi-0}, the occurrence set is $\{5, 2\}$ and its position set is $\{6, 9\}$. In a sense, one can say that being at node $\overline{issi}$ is being at positions 6 and~9 simultaneously. Note that for each node, the tree below it is the Patricia tree of the suffixes starting at the positions in its position set.

In a leaf, the occurrence set is a singleton, and we label the leaf by its element.

\begin{lemma}[Position set at the suffix node]\label{lemma:suffix-error}
Given two nodes $\overline{aw}$ and $\overline{w}$ (connected by a suffix link, labeled $a$), if one takes position set of $\overline{w}$, subtracts one to each element, one obtains a superset of the position set of $\overline{w}$. The items shared by both sets are those positions of the string which contain an $a$.
\end{lemma}

The lemma is fairly obvious given that the position set of $\overline{aw}$ contains all the positions where $aw$ occurs which will be exactly one minus those positions where $w$ occurs following $a$.

\begin{definition}[Error Tree]
For any node $w$ its error tree is the $\Sigma^+$ tree formed by taking it's position set, adding one to each element and forming the $\Sigma^+$ tree of the suffixes of the suffixes starting at those positions. If the position set includes the end of the string, that element is removed.

The leafs are labeled by the position of the string in which their paths occur minus $|w| + 1$.
\end{definition}

For example, in the above mentioned node $\overline{issi}$, the error tree is formed by taking the strings starting at positions $\{7, 10\}$ (ie \putstring{sippi\$} and \putstring{i\$}) in a Patricia tree.

\begin{definition}[1-error dotted Tree]
We define a \emph{1-error dotted tree} as the tree which is formed by adding to each node in a suffix tree, a new edge labeled by \putstring{.} which points to its error tree. The edge labeled \putstring{.} shall be called a \emph{dot link}.
\end{definition}

\begin{figure} 
\includegraphics[width=\textwidth]{figures/mississippi-1.eps}
% mississippi-1.eps: 300dpi, width=6.08cm, height=1.63cm, bb=0 0 718 193
\caption{1-error dotted tree for \putstring{mississippi}}%
\label{fig:mississippi-1}
\end{figure}


The one-error dotted tree for \putstring{mississippi} is shown in Figure~\ref{fig:mississippi-1}. The nodes are connected to their error trees by thick diagonal links.

The paths in the dotted tree are paths in the extended alphabet $\Sigma\cup\{.,\$\}$. The notions of \emph{occurrence set}, \emph{position set} and \emph{error tree} are valid for all nodes in a dotted tree considering that a node path may contain dot elements which match any other character.

\begin{definition}[k-error dotted tree]
We define a k-error dotted tree as the tree obtained by adding error trees to each node in the (k-1)-error dotted tree which does not already contain one.
\end{definition}


\section{Space Considerations}
\begin{figure}
\centering
\includegraphics[height=.2\textheight]{figures/aaaa}
\caption{$a^n$: worst case space consumption}%
\label{fig:aaaa}
\end{figure}

How much space does a k-error dotted tree take? Suffix Trees have a number of nodes proportional to the size of the string (which we will refer to as $n$). One-error dotted trees have more nodes. The worst case presented in Figure~\ref{fig:aaaa} is a string of the form $a^n = aaa\cdots$ which generates a one-dotted tree taking $\mathcal{O}(n^2)$ nodes. For a larger number of errors $k$, it is easy to see that this leads to $\mathcal{O}(n^{k+1})$ nodes.

We can do better, however, by considering the expected case.

\begin{definition}
We define $N_k$ to be the number of nodes of the k-error dotted tree.

$N_0$ is the number of nodes of the suffix tree.
\end{definition}

\begin{definition}[Maximum Path Lenght]
$l$ is the length of the deepest node in the tree (which isn't a leaf)

\[ l = \max_{\overline{w} \in \mathcal{T}} |w| \mathrm{and \overline{w} isn't a leaf} \]
\end{definition}

\begin{definition}[Suffix-Depth]
For any node $\overline{w}$ its suffix-depth is the number of nodes which must be followed to reach the root.
\end{definition}

For a node $\overline{w}$, its suffix-depth is $|w|$. It is obvious that the $l$ is a maximum on the suffix-depth of nodes.

We can now show that $N_k = \mathcal{O}(nl^k)$ by induction. It is a known fact that $N_0 = \mathcal{O}(n)$. The algorithm for turning a k-error into a (k+1)-error dotted tree, can be looked at the following way~\footnote{We are here considering that the \textit{foreach} loop of Figure~\ref{algo:construct-k} runs in a specific order which is in fact difficult to code for. However, as an analysis tool, it is a valid assumption.}: First it makes an almost exact copy of the k-error dotted tree and sets it as the root's error tree and it clears all the other error trees. Then it proceeds in stages, making a (possibly incomplete) copy of this tree spread amongst the nodes at suffix-depth 1. It proceeds to the nodes at suffix-depth 2 and on to increasing suffix-depths up to $l$. At each such step, the number of nodes is increased by a maximum of $N_k$. Therefore, we start with $N_k$ nodes, make an almost full copy, and copy that at most $l$ times. We have $N_{k+1}=\mathcal{O}(N_k(l+1))$. Assuming $N_k=\mathcal{O}(nl^k)$ by induction we conclude $N_{k+1}=\mathcal{O}(nl^{k+1})$.

So far, we have achieved little since in the worst case $l=n-1$ as shown above. However, the expected case is $l=\mathcal{O}(\log n)$~\cite{apostolico92selfalignments} and we have $N_k=\mathcal{O}(n\log^k n)$.


\section{Searching}
To search a string with at maximum k errors in a k-error dotted tree, we descend the tree. Whenever we find a node, we follow both the dot link and the edge which matches the current position of the string we are searching for. If we find a mismatch in the middle of an edge, we continue, decreasing the amount of errors permitted. When presenting the algorithm, we take the liberty of using $s+j$ where $s$ is a string and $j$ an integer to mean the suffix of $s$ starting at position $j$ (as happens in the C~Programming Language).

\begin{algorithm}
\caption{Search Algorithm: Function findString($\overline{w}$,$s$,$k$)}%
\label{algo:search}
\SetKwData{J}{j}
\SetKwData{W}{s}
\SetKwData{K}{k}
\SetKwData{Len}{len}
\SetKwData{Node}{node}
\SetKwData{Start}{nodeStart}
\SetKwData{String}{treeString}

\KwIn{Current node \Node}
\KwIn{String \W}
\KwIn{Maximum errors \K}
\KwData{The tree's string \String}

\If{\W is empty}{
	\emph{report all \Node's children}\;
}
\If{$\K > 0$}{
	findString(\Node.dotLink, $\W+1$,$\K-1$)\nllabel{algoline:search:followDot}\;
}
\Node \Assign \Node.getSon($\W_0$)\;
\Len \Assign min(length(\Node),length(\W))\;
\For{\J \Assign 1 \KwTo \Len}{
	\If{$\W_{\J} \not= \String_{\Start+\J}$}{
		\eIf{$\K > 0$}{
			\K \Assign $\K-1$\;
		}{
			\emph{return, the string was not found}\;
		}
	}
}
findString(\Node,$\W+\Len$,\K)
\end{algorithm}


The process just described is Algorithm~\ref{algo:search}. Reporting of the leafs below a certain node can be done by a Depth First Search. If one is interested in knowing only whether the string occurs, then the lines which report the matches can be substituted by early exits returning true.

This algorithm runs in time $\mathcal{O}(m^{k+1})$: There are at most $\sum_{i=0}{k}{m \choose i}=\mathcal{O}(m^k)$ different paths which must be followed. Following one path in the tree takes $\mathcal{O}(m)$ time. Therefore, the total time is $\mathcal{O}(m^{k+1})$ without the reporting of occurrences (which can be done in time linear to the number of occurrences).

\section{Constructing the Dotted Tree}
\subsection{One-Error Dotted Tree}

Constructing the tree is very simple. We start with a suffix tree which includes suffix links and add error trees to it. First, we construct the error tree for the root. This tree is almost a copy of the entire tree, except for two properties:

\begin{enumerate}
\item It does not have the leaf labeled 1 in the original tree. Keeping this leaf would have resulted in a leaf labeled \putstring{.s\$} which does not occur in original string (it is one character too long).
\item For any other leaf $\overline{w\$}$ occuring at position $p$ in the string, we have a new leaf $\overline{.w\$}$ which occurs at position $p-1$ in the string. Therefore, we adjust the labels.

\end{enumerate}

For any other node $\overline{aw}$, the error tree is a copy of the error tree at node $\overline{w}$ (the node pointed to by node $\overline{aw}$'s suffix link) with the following changes:

\begin{enumerate}
\item The leaf labeled 1 in the original error tree is not included.
\item Leafs in the copy have a label which is the original label's value minus one.
\item A leaf labeled $p$ is included only if $s_{p-1}$ equals the label of the suffix link.
\end{enumerate}

These conditions are an expression of Lemma~\ref{lemma:suffix-error} and an extension of the conditions for the root. Both are implemented by Algorithm~\ref{algo:copy-subtree}.

\begin{algorithm}
\caption{Copying a sub tree}\label{algo:copy-subtree}
\SetKwData{Copy}{copy}
\SetKwData{SonCopy}{copy of son}
\SetKwData{String}{string}

\KwIn{A node in a suffix tree $\overline{w}$}
\KwIn{An optional character $a$ (not given when copying the root)}
\KwData{The original string \String}

\Copy \Assign copy($\overline{w}$)\;
\If{$\overline{w}$ is a leaf}{
	$p$ \Assign $\overline{w}$.label\;
	\lIf{$p = 1$}{\Return null}\;
	\If{$a$ was not given or $\String_{p-1} = a$}{
		\Copy.label \Assign \Copy.label - 1\;
		\Return \Copy
	}
}
\ForEach{$n \in \overline{w}.\mathit{sons}$}{
	\SonCopy \Assign copySubtree($n$,$a$)\;
	\If{\SonCopy isn't null}{
		\Copy.sons \Assign \Copy.sons $\cup$ \SonCopy\;
	}
}
\lIf{\Copy.sons is empty}{ \Return null }\;
\If{\Copy.sons has only one element}{
	\nllabel{algoline:copy-subtree:merge}\emph{merge \Copy.sons into \Copy and return that}\;
}

\end{algorithm}


The only point to note is line~\ref{algoline:copy-subtree:merge}. Since we filter some leafs, without a merging procedure, it would be possible to have nodes which possess only one child. These are removed by merging a child with its parent.

\begin{figure}
\centering
\includegraphics[width=.4\textwidth]{merge}
\caption{Node Merging demonstration}%
\label{fig:merge}
\end{figure}

As discussed above, suffix trees are implemented such that each node is actually just a pair of indices into the original string (exemplified in Figure~\ref{fig:mississippi-nodes}). To merge a child with its parent, we just need to replace both by a node whose starting index is the starting index of the original child minus the length of the parent edge. This process is illustrated in Figure~\ref{fig:merge}.

The construction of the tree using either Ukonnen's or McCreight's algorithm assures that this operation is correct due to the way that the indices are created.

\begin{algorithm}
\dontprintsemicolon
\caption{addDotLink: add an error tree to a node}\label{algo:addDotLink}
\KwIn{A node $n$}
\lIf{\emph{already has dot tree}}{\Return}\;
addDotLink($n$.suffixLink)\;
$\overline{w}$.dotLink = copySubtree($\overline{w}$.suffixLink, \emph{ignore $\overline{w}$}.suffixLinkLabel)\;
\end{algorithm}


One can construct the error tree for any node if the tree for the node pointed to by the current node's suffix link has an error tree. This leads to a recursive definition implemented as Algorithm~\ref{algo:addDotLink}. 

\begin{algorithm}
\dontprintsemicolon
\caption{Turn suffix tree into a dotted tree}\label{algo:construct-1}

\KwIn{A suffix tree $\mathcal{T}$}

Root.dotLink = copySubtree(Root)\;
\nllabel{algoline:construct-1:foreach}\ForEach{$\overline{w} \in \mathcal{T}$}{
	addDotLink($\overline{w}$)\;
}
\end{algorithm}


Algorithm~~\ref{algo:construct-1} constructs the 1-error dotted tree. The \textit{foreach} loop at line~\ref{algoline:construct-1:foreach} can be implemented using Depth First Search. In fact, the procedure addDotLink only needs to be called for certain nodes (those which do not possess an incoming suffix link). However, there is no readily available way to access only these nodes and the procedure above does little extra work.

\subsubsection*{Time cost}

If the number of nodes in the final tree is $N$ and the alphabet $\Sigma$, then the above algorithm runs in time $\mathcal{O}(N|\Sigma|)$.

The error tree at the root is created in time proportional to the number of nodes it contains. Every other error tree is constructed by looking at an existing one and copying it. Each of these copies need time proportional to the original error tree. Since there are at maximum $|\Sigma|$ incoming suffix links to a node, each error tree is \emph{looked at} a maximum of $|\Sigma|$ times. The sum of all these operations is therefore bounded by $|\Sigma|N$.

\subsection{Extension to any number of errors}\label{subsec:construct-k-larger-1}

\begin{algorithm}
\dontprintsemicolon
\caption{Copying a sub tree}\label{algo:copy-subtree-with-dot-link}
\SetKwData{Copy}{copy}
\SetKwData{SonCopy}{copy of son}
\SetKwData{String}{string}

\KwIn{A node in a suffix tree $\overline{w}$}
\KwIn{An optional character $a$ (not given when copying the root)}
\KwData{The original string \String}

\Copy \Assign make-copy($\overline{w}$)\;
\If{$\overline{w}$ is a leaf}{
	$p$ \Assign $\overline{w}$.label\;
	\lIf{$p = 1$}{\Return null}\;
	\If{$a$ was not given or $\String_{p-1} = a$}{
		\Copy.label \Assign \Copy.label - 1\;
		\Return \Copy
	}%
}%
\Copy.dotLink \Assign copySubtree($\overline{w}$.dotLink,$a$)\;
\ForEach{$n \in \overline{w}.\mathit{sons}$}{
	\SonCopy \Assign copySubtree($n$,$a$)\;
	\If{\SonCopy isn't null}{
		\Copy.sons \Assign \Copy.sons $\cup$ \SonCopy\;
	}
}
\lIf{\Copy.sons is empty}{ \Return null }\;
\If{\Copy.sons has only one element}{
	\emph{merge \Copy.sons into \Copy and return that}\;
}

\end{algorithm}


The above algorithm can be used to construct trees with any number of errors by iterating it. To construct the (k+1)-error tree from the k-error tree, make an adjusted copy of the tree as above (adjusting leafs and filtering the leafs labeled one) and make this the new root error tree. Then, for every other node, remove the current error tree (it has a level too few). Finally, for every node except the root, construct its error tree as above (adjusting leafs, filtering the leafs labeled one or which do not correspond to positions in the suffix link's label). These error trees will now have the correct strings.

\begin{algorithm}
\caption{Algorithm for turning a suffix tree into a dotted tree}\label{algo:construct-k}
\SetKwData{K}{k}
\SetKwData{Null}{null}

\KwIn{A suffix tree $\mathcal{T}$}
\KwIn{An integer \K}

\SetKwBlock{DoKTimes}{Repeat \K Times}{}
\DoKTimes{
	Root.dotLink = copySubtree(Root)\;
	\ForEach{$\overline{w} \in \mathcal{T}$}{
		$\overline{w}$.dotLink \Assign \Null\;
	}
	\ForEach{$\overline{w} \in \mathcal{T}$}{
		addDotLink($\overline{w}$)\;
	}
}
\end{algorithm}


The code for the subtree copy is Algorithm~\ref{algo:copy-subtree-with-dot-link} which differs from Algorithm~\ref{algo:copy-subtree} in that dot link are also followed while copying the tree. The final procedure for constructing k-error trees is Algorithm~\ref{algo:construct-k}.

We note that the method of removing the existing error trees can be improved in practice (the asymptotic bound remains the same) the following way: While making a copy of the tree for construction of the error tree at the root, instead of following a dot link to make a copy, move the error tree to its destination and make the necessary adjustments in place (this move will probably be just an adjustment of pointer depending on the exact implementation). Using this method only leafs labeled one need to be removed (as well as any internal node which becomes unnecessary).

\begin{definition}
We define $N_k$ to be the number of nodes of the k-error dotted tree. $N_0$ is the number of nodes of the suffix tree. If $k$ is known from context we write $N$ for $N_k$.
\end{definition}

The analysis for the time cost above remains valid. We now have that the time cost is $\mathcal{O}(N_1|\Sigma|+N_2|\Sigma|+\dots+N_k|\Sigma)=\mathcal{O}((N_1+N_2+\dots+N_k)|\Sigma|)=\mathcal{O}(kN|\Sigma|)$.


\section{Experimental Results}
\subsection{Implementation and Data Sets}

The code was implemented in the \CC~Language~\footnote{Code is available upon request.}. Three data sets were used: a set of English texts; the DNA of S. Cerevesiae; and randomly generated text (uniform distribution). For testing we used an Intel Pentium~IV running at~2.40GHz with~4GB of main memory.

\subsection{Index Size}

\begin{figure*}%
\centering
\subfigure[DNA]{\includegraphics[height=3cm,width=.45\textwidth]{ratio-in=dna-01-12.eps}}%
\subfigure[English]{\includegraphics[height=3cm,width=.45\textwidth]{ratio-in=english-01-12.eps}}\\%
\subfigure[Random]{\includegraphics[height=3cm,width=.45\textwidth]{ratio-in=random-01-12.eps}}%
\caption{Ratio of number of nodes in k-error and (k+1)-error dotted trees}\label{fig:ratios}%
\end{figure*}
%
To experimentally verify the average case prediction, we show in Figures~\ref{fig:ratios} the ratios between the $k$-error and the $(k+1)$-error dotted trees, for all datasets. We can easily see that the experimental values do resemble a logarithm as predicted.
%
\subsection{Search Time}
%
\begin{figure*}
\centering
\subfigure[k=1]{\includegraphics[width=.75\textwidth,height=5cm]{search-vary-N.in=all.k=1.steps.eps}}\\
\subfigure[k=2]{\includegraphics[width=.75\textwidth,height=5cm]{search-vary-N.in=all.k=2.steps.eps}\hspace{-1cm}}
\caption{Search time versus text size}\label{fig:search-vary-N}
\end{figure*}
%
For searching, the texts were first indexed and then searches were performed on top of the structure. The search algorithm performed an early exit, without reporting of occurrences (ie, it only reported whether the string exists in the text). Therefore, the number of occurrences had no influence on the search time.
%
%Figure~\ref{fig:search-vary-m} shows the search time for growing pattern lengths. It shows the case where the pattern exists and where it does not (an extra error is added). The fact that the search is done using an early-exit strategy explains why non existing patterns take longer than existing patterns. The value presented is the number of character comparisons the algorithm makes.

Figure~\ref{fig:search-vary-N} shows searching for a 15~character long sequences on a $k$-error dotted tree. Presented are the number of steps in the algorithm (averaged over 10\,000 such searches) while varying the text size. We see that after an initial small growth which is explained by the increasing density of the tree, the search time is roughly constant.

\subsection{Varying Alphabet Size}
%
\begin{figure*}
\centering
\subfigure[Nodes]{\includegraphics[width=.35\textwidth]{nodes-vary-E.eps}}
%\subfigure[Time]{\includegraphics[width=.25\textwidth]{construction-time-vary-E.eps}}
\subfigure[Ratio]{\label{subfig:nodes-vary-E-ratio}\includegraphics[width=.35\textwidth]{construction-time-ratio-vary-E.eps}}
\caption{Construction time versus $|\Sigma|$}\label{fig:nodes-vary-E}
\end{figure*}%
%
\begin{figure*}
\centering
\subfigure[Time]{\includegraphics[width=.35\textwidth]{search-time-vary-E.eps}}%
\subfigure[Steps]{\includegraphics[width=.35\textwidth]{search-steps-vary-E.eps}}
\caption{Searching versus $|\Sigma|$}\label{fig:search-vary-E}
\end{figure*}%
%
We generated 100\,KB long strings using a uniform distribution. We varied the alphabet from binary through 100 and constructed a~2-error tree. As above, searches were for 15~character long strings, averaged over 10\,000 repetitions.

As predicted, the time to construct one node in the tree (shown in Figure~\ref{subfig:nodes-vary-E-ratio}) grows in a roughly linear fashion. The search time (shown in Figure~\ref{fig:search-vary-E}) does not grow significantly with alphabet size.

\section{Future Work, Open Problems}

In the example for the string \putstring{mississipi}, presented in Figure~\ref{fig:mississipi-1}, one can see that the tree below \putstring{s.i} and \putstring{ssi} are exactly the same. Whether such occurences are the basis for a significant space saving and what algorithms mightexploit them is an open question.
Along the same lines, the structure's definition might be extended to structures such as the suffix-DAG presented in~\cite[7.7]{gusfield:algorithms}. An algorithm to efficiently construct it over these suffix-DAGs is another open problem.

\bibliographystyle{alpha}
\bibliography{biblio}
\end{document}
