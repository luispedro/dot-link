\documentclass[a4paper,10pt]{article}
\usepackage[boxruled,vlined,portugues,figure,linesnumbered,longend]{algorithm2e}
\input{figures/transfig}
\usepackage[]{graphicx}
\usepackage{amssymb}
\usepackage{QED}


\newcommand*{\Assign}{\ensuremath\longleftarrow}
\newcommand{\putstring}[1]{\textsl{#1}}
\newtheorem{definition}{Definition}
\newtheorem{lemma}{Lemma}
\graphicspath{{figures/}}

\title{Dotted Suffix Trees\\A Structure for Approximate Text Indexing}
\author{L. P. Coelho \and A. Oliveira}

\begin{document}

\maketitle

\begin{abstract}
The problem we address is text indexing for approximate matching. We consider that we are given a text $\mathcal{T}$ which undergoes some preprocessing to generate an index. We can later query this index to identify the places where a string occurs up to a certain number of allowed errors~$k$ (edition distance). We present a structure for indexing which occupies space $\mathcal{O}(n\log^kn)$ in the average case, independent of alphabet size, $n$ being the text size. This structure can be used to report the existence of a match with $k$ errors in $\mathcal{O}(3^k m^{k+1})$ and to report the occurrences in $\mathcal{O}(3^k m^{k+1} + \mbox{\it ed})$ time, where $m$ is the length of the pattern and where {\it ed} the number of matching edit scripts. These bounds are independent of alphabet size. The construction of the structure has time bound by $\mathcal{O}(kN|\Sigma|)$, where $N$ is the number of nodes in the index and $|\Sigma|$ the alphabet size. 

\textbf{Keywords:} string algorithms, suffix trees, approximate text matching, text indexing.

\end{abstract}

\section{Introduction}\label{sec:introduction}
Since their introduction~\cite{DBLP:conf/focs/Weiner73}, suffix trees have been one of the methods of choice for text indexing. Suffix trees solve the following problem: given a text which has been preprocessed, and a pattern, efficiently find the places in the text where this pattern occurs.  However, for many real-life problems this is too restrictive. In this class of problems, one is interested in finding places in the text where an approximate form of the pattern occurs. Several algorithms have been proposed to solve this problem. \cite{navarro:guided} presents a survey of the problem and existing approaches to solving it at the time that work was written. Recently there have been some further work in the field, notably~\cite{maass:text} which also contains an updated overview of methods for solving the problem in addition to presenting one itself.

In this work we present an approach which solves this problem based on an extension of suffix trees. The main advantage of this approach is that both the search and the index size are \emph{alphabet independent}, although the indexing time is not.

\subsection{Related Work}

Our work is related to efforts to index the text for approximate matching such as~\cite{maass:text,amir00oneerror}. The structure presented is superficially very similar to the one presented in~\cite{DBLP:journals/tcs/ChattarajP05} as an \textit{inexact suffix tree} though that work is concerned with solving a different set of problems and the algorithms presented are fundamentally different.

\subsection{Structure of the Paper}

In section~\ref{sec:structure} we present the structure and the way to use it for searching. In section~\ref{sec:build-1} we show the way to build a structure to handle one error only and analyze this version of the algorithm. Then, we show, in section~\ref{sec:build-k} how to generalize this procedure for any number of errors and how the previous analysis also applies to any number of errors.



\section{The Indexing Structure}\label{sec:structure}
\begin{figure*}
\centering
\includegraphics[width=.6\textwidth]{figures/mississippi-0}
\caption{Suffix tree representation for the string \putstring{mississippi}}%
\label{fig:mississippi-0}
\end{figure*}
\subsection{Intuition}
Suffix Trees have become a well known structure since their introduction in 1973~\cite{DBLP:conf/focs/Weiner73}. The example in Figure~\ref{fig:mississippi-0} shows the suffix tree for the string \putstring{mississippi}.

To extend the suffix tree, we consider that a node in the suffix tree corresponds to several points in the string simultaneously (those points where the string on the path to the node occurs). To each node we add an extra edge corresponding to moving one character forward in all these points.

\subsection{Formal definition}

We assume that the reader is familiar with the basic concepts of suffix trees and strings and present the definitions below mainly to introduce the particular notation and terminology used in this paper.

\begin{definition}[Character, string]
Given a set $\Sigma$, we say that $s$ is a \emph{string over $\Sigma$} if $s$ is a (possibly empty) sequence of elements of $\Sigma$. The elements of the set $\Sigma$ are also called \emph{characters}. The length of the string, denoted by $|s|$ is the number of elements it contains. We shall write $s_i$ for the $i$th element of the string. For the empty string, we write $\varepsilon$.

The set of all strings is denoted by $\Sigma^*$ and $\Sigma^+=\Sigma^*-\{\varepsilon\}$.
\end{definition}

For denoting characters we shall use letters from the beginning of the roman alphabet ($a$, $b$, $c$,\ldots) and for strings, we shall use letters from the end of the alphabet ($w$, $z$, \ldots). In what follows we assume that there are two special symbols (\$ and $.$) which are not part of $\Sigma$.

\begin{definition}[Concatenation]
$wx$ or $aw$ will denote the usual concatenation operation.

If $s = wxy$, then $w$ is a \emph{prefix} of $s$, $x$ is a \emph{substring} of $s$ (at position $|w|$) and $y$ is a \emph{suffix} of $s$ (at position $|wx|$).
\end{definition}

\begin{definition}[Patricia tree]
$T$ is a Patricia tree if $T$ is a rooted tree with edge labels from $\Sigma^+$. For each $a \in \Sigma$ and every node $n$ in $T$, there is at most one edge leaving $n$ whose label starts with $a$. Each node in a Patricia tree has a path leading to it which forms a string. If the node $n$ has the leading path $w$, we shall also refer to $n$ as $\overline{w}$.

A \emph{compact} Patricia tree is one with no nodes with just one child.
\end{definition}

%The name Patricia is sometimes spelled PATRICIA, since it was formed from the initial of Practical Algorithm to Retrieve Information Coded in Alphanumeric~\cite{DBLP:journals/jacm/Morrison68}.
%
In this work we are interested in approximate matching using edition (or Levenshtein) distance:

\begin{definition}[Aproximate Match]
We say that the string $s$ matches the string $t$ at position $p$ with $k$ errors if we can make $k$ modifications in $s$ to obtain $s'$ which is a substring of $t$ at position $p$. A modification is either deletion, insertion or substitution of one character.
\end{definition}

\begin{definition}[Suffix Tree, Suffix Link]
A \emph{suffix tree} for a string $S$ is a compact Patricia tree whose leaf nodes (those without children) have paths corresponding to all suffixes of the string $S\$ $. A suffix link in a suffix tree is a link from the node $\overline{aw}$ to the node $\overline{w}$. This link has the label $a$.
\end{definition}

In a suffix tree, all internal nodes have a well defined suffix link. Both Ukkonen's~\cite{ukkonen92constructing} and McCreight's~\cite{mccreight} algorithms are linear time algorithms for constructing a suffix tree which includes suffix links. The suffix links are shown in Figure~\ref{fig:mississippi-0} as dashed arrows with the label in italic.

Suffix trees are normally implemented by associating with each node two indices into the original string which are the start and end position of its incoming edge. This allows the structure to be kept in $\bigO(n)$ space.

\begin{definition}[Point in a suffix tree]
A \emph{point} in a suffix tree is either a node in the tree or a point in the middle of an edge.
\end{definition}

\begin{definition}[Occurrence Set, Position Set]
Given a node $\overline{w}$ in a suffix tree, we call its \emph{occurrence set} the set of indexes in the original string where the string $w$ occurs.

Given a node $\overline{w}$ in a suffix tree, its \emph{position set} is the set formed by taking its occurrence set and adding the length of $w$ to each element.
\end{definition}

For example, in the node $\overline{issi}$ in the suffix tree of Figure~\ref{fig:mississippi-0}, the occurrence set is $\{2, 5\}$ and its position set is $\{6, 9\}$. In a sense, one can say that being at node $\overline{issi}$ is being at positions 6 and~9 simultaneously. For each node, the tree below it is the Patricia tree of the suffixes starting at the positions in its position set. In a leaf, the occurrence set is a singleton, and we label the leaf by its element.

\begin{lemma}[Position set at the suffix node]\label{lemma:suffix-error}
Given two nodes $\overline{aw}$ and $\overline{w}$ (connected by a suffix link, labeled $a$), if one takes position set of $\overline{w}$, subtracts one from each element, one obtains a superset of the position set of $\overline{w}$. The items shared by both sets are those positions of the string which contain an $a$.
\end{lemma}

The lemma is fairly obvious given that the position set of $\overline{aw}$ contains all the positions where $aw$ occurs which are exactly those positions where $w$ occurs preceded by $a$.

\begin{definition}[Error Tree]
For any node $w$, its error tree is the Patricia tree formed by taking its position set, adding one to each element and forming the Patricia tree of the suffixes starting at those positions. If the position set includes the end of the string, that element is removed.

The leaves are labeled by the position of the string in which their paths occur minus $|w| + 1$.
\end{definition}

For example, in the above mentioned node $\overline{issi}$, the error tree is formed by taking the strings starting at positions $\{7, 10\}$ (ie \putstring{sippi\$} and \putstring{pi\$}) in a Patricia tree.

\begin{definition}[1-error dotted Tree]
We define a \emph{1-error dotted tree} as the tree which is formed by adding to each node in a suffix tree, a new edge labeled by \putstring{$\cdot$} which points to its error tree. The edge labeled \putstring{$\cdot$} shall be called a \emph{dot link}.
\end{definition}

\begin{figure*}
\centering
\includegraphics[width=\textwidth]{figures/mississippi-1.eps}%
\caption{1-error dotted tree for \putstring{mississippi}}%
\label{fig:mississippi-1}%
\end{figure*}%
The 1-error dotted tree for \putstring{mississippi} is shown in Figure~\ref{fig:mississippi-1}. The nodes are connected to their error trees by thick diagonal links.

The paths in the dotted tree are paths in the extended alphabet $\Sigma\cup\{.,\$\}$. The notions of \emph{occurrence set}, \emph{position set} and \emph{error tree} are valid for all nodes in a dotted tree.

\begin{definition}[k-error dotted tree]
We define a $k$-error dotted tree as the tree obtained by adding error trees to each node in the $(k-1)$-error dotted tree which does not already contain one.
\end{definition}



\section{Space Considerations}\label{sec:space}
\begin{figure}
\includegraphics{figures/aaaa}
\caption{$a^n$: worst case space consumption}%
\label{fig:aaaa}
\end{figure}

How much space does a k-error dotted tree take? Suffix Trees have a number of nodes proportional to the size of the string (which we will refer to as $n$). One-error dotted trees have more nodes. The worst case presented in Figure~\ref{fig:aaaa} is a string of the form $a^n = aaa\cdots$ which generates a one-dotted tree taking $\mathcal{O}(n^2)$ nodes. For a larger number of errors $k$, it is easy to see that this leads to $\mathcal{O}(n^{k+1})$ nodes.

We can do better, however, by considering the expected case.

\begin{definition}
We define $N_k$ to be the number of nodes of the k-error dotted tree.

$N_0$ is the number of nodes of the suffix tree.
\end{definition}

\begin{definition}[Maximum Path Lenght]
$l$ is the lenght of the deepest node in the tree (which isn't a leaf)

\[ l = \max_{\overline{w} \in \mathcal{T}} |w| \mathrm{and \overline{w} isn't a leaf} \]
\end{definition}

\begin{definition}[Suffix-Depth]
For any node $\overline{w}$ its suffix-depth is the number of nodes which must be followed to reach the root.
\end{definition}

For a node $\overline{w}$, its suffix-depth is $|w|$. It is obvious that the $l$ is a maximum on the suffix-depth of nodes.

We can now show that $N_k = \mathcal{O}(nl^k)$ by induction. It is a known fact that $N_0 = \mathcal{O}(n)$. The algorithm for turning a k-error into a (k+1)-error dotted tree, can be looked at the following way~\footnote{We are here considering that the \textit{foreach} loop of Figure~\ref{algo:construct-k} runs in a specific order which is in fact difficult to code for. However, as an analysis tool, it is a valid assumption.}: First it makes an almost exact copy of the k-error dotted tree and sets it as the root's error tree and it clears all the other error trees. Then it makes a (possibly incomplete copy) broken amongst all the nodes whose suffix links point at the root (nodes at suffix-depth 1). It proceeds to the nodes at suffix-depth 2 and on to increasing suffix-depths up to $l$. At each such step, the number of nodes is increased by a maximum of $N_k$. Therefore, we start with $N_k$ nodes, make an almost full copy, and copy that at most $l$ times. We have $N_{k+1}=\mathcal{O}(N_kl)$. Assuming $N_k=\mathcal{O}(nl^k)$ by induction we conclude $N_{k+1}=\mathcal{O}(nl^{k+1})$.

So far, we have achieved little since in the worst case $l=n-1$ as shown above. However, the expected case is $l=\mathcal{O}(\log n)$~\cite{apostolico92selfalignments} and we have $N_k=\mathcal{O}(n\log^k n)$.



\section{Searching}\label{sec:search}
Given a pattern to search for, we descend the tree point by point, following the pattern character by character. We represent this by keeping a node and an offset from the start of its incoming link. In an edge, we consider that there is an implicit dot link which goes forward one character. At each point, we can take four possible actions:

\begin{description}
\item[match] Where we descend the tree according to the pattern (may not be possible).
\item[substitution] We follow the dot link (possibly implicit), moving in the pattern.
\item[insertion] We follow the (possibly implicit) dot link, while not moving in the pattern.
\item[deletion] We advance in the pattern, while not moving in the tree.
\end{description}

\begin{algorithm}
\dontprintsemicolon
\SetKwData{J}{j}
\SetKwData{W}{s}
\SetKwData{K}{k}
\SetKwData{Len}{len}
\SetKwData{Offset}{offset}
\SetKwData{Child}{child}
\SetKwData{Start}{nodeStart}
\SetKwData{String}{treeString}
\caption{Function findString(\Node, offset, $s$, $k$)}%
\label{algo:search}

\KwIn{Current node \Node}
\KwIn{Current offset \Offset}
\KwIn{String \W}
\KwIn{Maximum errors \K}
\KwData{The tree's string \String}

\lIf{$\K < 0$}{return \emph{string not found}\;}
\lIf{\W is empty}{
	\emph{report all \Node's children}\;
}
findString(\Node,\Offset,$\W+1$,$\K-1$)\tcp{deletion}\;
\eIf{$\Offset = \mathrm{length}(\Node)$}{
	findString(\Node.dotLink, 0, $\W$,$\K-1$)\tcp{insertion}\;
	findString(\Node.dotLink, 0, $\W+1$,$\K-1$)\tcp{substituition}\nllabel{algoline:search:followDot}\;
	\Child \Assign \Node.getSon($\W_0$)\tcp{try matching}\;
	\lIf{\Child isn't null}{findString(\Child, 0, $\W+1$,$\K$)\;}
}{
	findString(\Node,\Offset+1,$\W$,\K-1)\tcp{insertion}\;
	\lIf{$\W_0 \not= \String_{\mathrm{start}(\Node)+\Offset}$}
		{ \K \Assign $\K-1$\;}
	findString(\Node,\Offset+1,$\W+1$,\K)\tcp{either match or substituition}\;
}
\end{algorithm}

%
Going down the tree can therefore be seen as a sequence of edition and matching operations: For example, for the string \putstring{ixsi}: \operation{match(i)}, \operation{substitution(x)}, \operation{match(s)}, \operation{match(i)}, where there are at most $k$ elements which are not \operation{match(c)}. There are at most $\sum_{i=1}^k{m \choose i}=\mathcal{O}(m^k)$ ways to combine $k$ edit operations into a string of size $m$. Since there are $3$ different operations (substitution, insertion, and deletion), we have at most $\mathcal{O}(3^k m^k)$ different edition sequences. Each sequence has at most $m+k=\bigO(m)$ (given that $k<m$) elements and therefore the total time to find matches is $\mathcal{O}(3^k m^{k+1})$. Once a match has been found in the tree, a Depth First Search can be used to report all the leaves below the node. This takes time proportional to the number of leaves, ie.\ proportional to the number of edit scripts which can be used to match the pattern to a substring of the text (which can be greater than the number of occurences~\footnote{As often happens, strings of the form $a^m$ serve as examples of pathological behaviour as they can match any position of a string of form $a^n$ in a large number of ways.}). The total search time is therefore $\bigO(3^k m^{k+1} + \mbox{\it ed})$.

The code needs to distinguish whether we are at the end of a branch (ie.\ at a node) or in the middle of one. When presenting the algorithm, we take the liberty of using $s+j$ where $s$ is a string and $j$ an integer to mean the suffix of $s$ starting at position $j$ (as happens in the C~Programming Language). The process just described is Algorithm~\ref{algo:search}.



\section{Constructing the Dotted Tree}\label{sec:construction}
\subsection{One-Error Dotted Tree}

Starting with a suffix tree which includes suffix links we add error trees to it. First, we construct the error tree for the root which is almost a copy of the entire tree, except for two properties:

\begin{enumerate}
\item It does not have the leaf labeled 1 in the original tree. Keeping this leaf would have resulted in a leaf labeled \putstring{.s\$} which does not occur in original string (it is one character too long).
\item For any other leaf $\overline{w\$}$ occurring at position $p$ in the string, we have a new leaf $\overline{.w\$}$ which occurs at position $p-1$ in the string. Therefore, we adjust the labels.
\end{enumerate}

For any other node $\overline{aw}$, the error tree is a copy of the error tree at node $\overline{w}$ (the node pointed to by node $\overline{aw}$'s suffix link) with the following changes:

\begin{enumerate}
\item The leaf labeled 1 in the original error tree is not included.
\item Leafs in the copy have a label which is the original value minus one.
\item A leaf labeled $p$ is included only if $s_{p-1} = a$ ($a$ being the label of the suffix link).
\end{enumerate}

\begin{algorithm}
\caption{Algorithm for copying a sub tree}\label{algo:copy-subtree}
\SetKwData{Copy}{copy}
\SetKwData{SonCopy}{copy of son}
\SetKwData{String}{string}

\KwIn{A node in a suffix tree $\overline{w}$}
\KwIn{An optional character $a$}
\SetKwData{String}{string}

\If{$\overline{w}$ is a leaf}{
	$p$ \Assign $\overline{w}$.label\;
	\eIf{$p = 1$} \Return{null}\;
	\If{$a$ was not given or $\String_{p-1} = a$}{
		\Copy \Assign $\overline{w}$\;
		\Copy.label \Assign \Copy.label - 1\;
		\Return{\Copy}
	}
}
\Copy \Assign copy($\overline{w}$)\;
\ForEach{$n \in \overline{w}.\mathit{sons}$}{
	\SonCopy \Assign copySubtree($n$,$a$)\;
	\If{\SonCopy isn't null}{
		\Copy.sons \Assign \Copy.sons $\cup$ \SonCopy\;
	}
}
\If{\Copy.sons is empty}{
	\Return{null}
}
\If{\Copy.sons has only one element}{
	\emph{merge \Copy.sons into \Copy and return that}\;
}

\end{algorithm}

\begin{algorithm}
\caption{addDotLink: add an error tree to a node}\label{algo:addDotLink}
\KwIn{A node $n$}
\If{\emph{already has dot tree}}{\Return{}}
addDotLink($n$.suffixLink)\;
$\overline{w}$.dotLink = copySubtree($\overline{w}$.suffixLink, \emph{ignore $\overline{w}$.suffixLinkLabel})\;
\end{algorithm}
%

These conditions are an expression of Lemma~\ref{lemma:suffix-error} and an extension of the conditions for the root. Both are implemented by Algorithm~\ref{algo:copy-subtree}. The only point to note is line~\ref{algoline:copy-subtree:merge}. Since we filter some leaves, without a merging procedure, it would be possible to have nodes which possess only one child. These are removed by merging a child with its parent.
\begin{figure}
\centering
\includegraphics[width=.4\textwidth]{merge}
\caption{Node Merging example}%
\label{fig:merge}
\end{figure}

As discussed above, suffix trees are implemented in a way such that each node stores the start and end indices of the substring on its incoming edge. To merge a child with its parent, we just need to replace both by a node whose starting index is the starting index of the original child minus the length of the parent edge. The construction of the tree using either Ukonnen's or McCreight's algorithm assures that this operation is correct due to the way that the indices are created.

One can construct the error tree for any node if the tree for the node pointed to by the current node's suffix link has an error tree. This leads to a recursive definition implemented as Algorithm~\ref{algo:addDotLink}. This procedure needs to be called for all nodes, which can be done using a simple depth first search on the tree.

\subsubsection*{Time Cost}

Copying a tree takes time proportional to the number of nodes it contains. The error tree at the root is a straightforward copy of the whole tree and every other tree is a copy of an existing one. Since each node can have at most $|\Sigma|$ incoming suffix links, each error tree is \emph{looked at\/} to be copied at most $|\Sigma|$ times. The sum of all these operations is therefore bounded by $|\Sigma|N$. Therefore, if the number of nodes in the final tree is $N$, then the above algorithm runs in time $\mathcal{O}(N|\Sigma|)$.

\subsection{Extension to any number of errors}\label{subsec:construct-k-larger-1}

\begin{algorithm}
\dontprintsemicolon
\caption{Turn a suffix tree into a dotted tree}\label{algo:construct-k}
\SetKwData{K}{k}
\SetKwData{Null}{null}

\KwIn{A suffix tree $\mathcal{T}$}
\KwIn{The maximum number of errors \K}

\SetKwBlock{DoKTimes}{Repeat \K Times}{}
\DoKTimes{
	Root.dotLink = copySubtree(Root)\;
	\ForEach{$\overline{w} \in \mathcal{T}$}{
		$\overline{w}$.dotLink \Assign \Null
	}
	\ForEach{$\overline{w} \in \mathcal{T}$}{
		addDotLink($\overline{w}$)
	}
}{}
\end{algorithm}

The above algorithm can be used to construct trees with any number of errors by iterating it. To construct the $(k+1)$-error tree from the $k$-error tree, make an adjusted copy of the tree as above (adjusting leaves and filtering the leaves with label~1) and make this the new root error tree. Then, for every other node, remove the current error tree (it has a level too few). Finally, for every node except the root, construct its error tree as above (adjusting leaves, filtering the leaves with label~1 or which do not correspond to positions in the suffix link's label). These error trees will now have the correct strings. The final procedure for constructing k-error trees is Algorithm~\ref{algo:construct-k}.

We note that the method of removing the existing error trees can be improved in practice (the asymptotic bound remains the same) the following way: While making a copy of the tree for construction of the error tree at the root, instead of following a dot link to make a copy, move the error tree to its destination and make the necessary adjustments in place.% (this move will probably be just an adjustment of pointer depending on the exact implementation). Using this method only leaves labeled one need to be removed (as well as any internal node which becomes unnecessary).

\begin{definition}
We define $N_k$ to be the number of nodes of the $k$-error dotted tree. $N_0$ is the number of nodes of the suffix tree. We will use $N$ for $N_k$ if $k$ is known from context.
\end{definition}

The analysis for the time cost above remains valid. We now have that the time cost is $\mathcal{O}(N_1|\Sigma|+N_2|\Sigma|+\dots+N_k|\Sigma)=\mathcal{O}((N_1+N_2+\dots+N_k)|\Sigma|)=\mathcal{O}(kN|\Sigma|)$.



\section{Experimental Results}\label{sec:experimental}
\subsection{Implementation and Data Sets}

The code was implemented in the C~Language, starting from the suffix tree implementation by FIXME.

Three data sets were used:

\begin{enumerate}
\item ``Democracy in America'' by Alexis de Tocqueville as natural language data ser
\item The DNA of S. Cervesiae
\item Randomly generated text
\end{enumerate}




\section{Future Work, Open Problems}\label{sec:future}

In the example for the string \putstring{mississipi}, presented in Figure~\ref{fig:mississipi-1}, one can see that the tree below \putstring{s.i} and \putstring{ssi} are exactly the same. Whether such occurrences are the basis for a significant space saving and what algorithms might exploit them is an open question.
Along the same lines, the structure's definition might be extended to structures such as the suffix-DAG presented in~\cite[7.7]{gusfield:algorithms}. An algorithm to efficiently construct it over these suffix-DAGs is another open problem.

\bibliographystyle{alpha}
\bibliography{biblio}
\end{document}
