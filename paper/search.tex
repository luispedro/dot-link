To search a string with at maximum k errors in a k-error dotted tree, we descend the tree. Whenever we find a node, we follow both the dot link and the edge which matches the current position of the string we are searching for. If we find a mismatch in the middle of an edge, we continue, decreasing the amount of errors permitted. When presenting the algorithm, we take the liberty of using $s+j$ where $s$ is a string and $j$ an integer to mean the suffix of $s$ starting at position $j$ (as happens in the C~Programming Language).

\begin{algorithm}
\caption{Search Algorithm: Function findString($\overline{w}$,$s$,$k$)}%
\label{algo:search}
\SetKwData{J}{j}
\SetKwData{W}{s}
\SetKwData{K}{k}
\SetKwData{Len}{len}
\SetKwData{Node}{node}
\SetKwData{Start}{nodeStart}
\SetKwData{String}{treeString}

\KwIn{Current node \Node}
\KwIn{String \W}
\KwIn{Maximum errors \K}
\KwData{The tree's string \String}

\If{\W is empty}{
	\emph{report all \Node's children}\;
}
\If{$\K > 0$}{
	findString(\Node.dotLink, $\W+1$,$\K-1$)\nllabel{algoline:search:followDot}\;
}
\Node \Assign \Node.getSon($\W_0$)\;
\Len \Assign min(length(\Node),length(\W))\;
\For{\J \Assign 1 \KwTo \Len}{
	\If{$\W_{\J} \not= \String_{\Start+\J}$}{
		\eIf{$\K > 0$}{
			\K \Assign $\K-1$\;
		}{
			\emph{return, the string was not found}\;
		}
	}
}
findString(\Node,$\W+\Len$,\K)
\end{algorithm}


The process just described is Algorithm~\ref{algo:search}. Reporting of the leafs below a certain node can be done by a Depth First Search. If one is interested in knowing only whether the string occurs, then the lines which report the matches can be substituted by early exits returning true.

This algorithm runs in time $\mathcal{O}(m^{k+1})$: There are at most $\sum_{i=0}^{k}{m \choose i}=\mathcal{O}(m^k)$ different paths which must be followed. Following one path in the tree takes $\mathcal{O}(m)$ time. Therefore, the total time is $\mathcal{O}(m^{k+1})$ without the reporting of occurrences (which can be done in time linear to the number of occurrences).
