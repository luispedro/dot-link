\begin{figure}
\includegraphics{figures/aaaa}
\caption{$a^n$: worst case space consumption}%
\label{fig:aaaa}
\end{figure}

How much space does a k-error dotted tree take? Suffix Trees have a number of nodes proportional to the size of the string (which we will refer to as $n$). One-error dotted trees have more nodes. The worst case presented in Figure~\ref{fig:aaaa} is a string of the form $a^n = aaa\cdots$ which generates a one-dotted tree taking $O(n^2)$ nodes. For a larger number of errors $k$, it is easy to see that this leads to $O(n^{k+1})$ nodes.

We can do better, however, by considering the expected case. First we need a pair of definitions:

\begin{definition}[weight under a node, weight of a tree]
The weight under a node $\mathit{Weight}(n)$ is the number of nodes in the subtree rooted at that node.

The total weight of a tree is the sum of the weights of all the nodes.
\end{definition}

Note that the weight of a tree \emph{is not the same} as the weight at the root.

\begin{definition}[height of a node, average height]
The height of a node $\mathit{Height}(n)$ is the number of nodes in the path leading to that node from the root. The average height of the tree $\hat{h}$, is the average height of its nodes.

\[ \hat{h} = \frac{1}{|\mathcal{T}|} \sum_{x \in \mathit{\mathcal{T}}} \mathit{Height}(x) \]

\[ h_{max} = \mathit{max}_{x \in \mathcal{T}} \mathit{Height}(x) \]

\end{definition}

\begin{lemma}
Given a tree (any tree, not just suffix or similar tree), the sum of the weights of all the nodes in the tree is equal to the average height of the node times the number of nodes.
\end{lemma}

\Proof
Consider the tree being built node by node, in any order, as long as the property that it remains a tree during the process is maintained. Each node added, adds a certain amount of weight to the tree. In particular it adds one for each node that sits above it in the tree. This is the same as saying that each node adds its height (which does not change). The weight of the final tree is the sum of all these values.

\[ \mathit{Weight}(\mathcal{T}) = \sum_{x \in \mathcal{T}}\mathit{Height}(x) = |\mathcal{T}|\frac{1}{|\mathcal{T}|}\sum_{x \in \mathcal{T}}\mathit{Height}(x) = |\mathcal{T}|\hat{h} \qed \]

\begin{definition}[Height of error tree]
Given an integer k, we define the height $h_k$ as the average height of all the error trees which exist in the k-error dotted tree and not in the (k-1)-error tree. $h_0$ is the average height of the original suffix tree.
\end{definition}

\begin{lemma}
The number of nodes of an k-error dotted tree is bounded by $O(nh_0h_1\dots{}h_{k-1})$.
\end{lemma}

\Proof
By induction. For zero errors, the number of nodes of a suffix tree is known to be $O(n)$~\cite{}. Taking a (k-1)-error dotted tree (which we assume, by induction, to have $O(nh_0\dots{}h_{k-2})$ nodes), we can add error trees to this to obtain a k-error dotted tree. These error trees will be added to the (k-1)-error trees, whose average height is $h_{k-1}$. The maximum total weight of these trees is the number of nodes times the average height, therefore $O(nh_0\dots{}h_{k-1})$. At each node, to which we add an error tree, that error tree has a number of nodes bounded by a multiple of the number of leafs of the node, which is in turn bounded by the weight at the node. The total number of added nodes is therefore bounded by the total weight $O(nh_0\dots{}h_{k-1})$ \qed.

So far, we have not really obtained much since, in the worst case, $h_i=O(n)$. However, these are all Patricia trees of suffixes of the original text, and the average expected case, is $h_i=O(\log n)$~\cite{}. This allows us to conclude that, in the average case, we will have $O(n\log^k n)$ nodes in our tree.

