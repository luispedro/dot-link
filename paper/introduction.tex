Since their introduction~\cite{weiner}, suffix trees have been one of the methods of choice for text indexing. They allow one to solve the following problem: given a text which has been preprocessed, and a pattern, efficiently find the places in the text where this pattern occurs.  However, for many real-life problems this statement is too restrictive. In these problems, one is interested in finding places in the text where an approximate form of the pattern occurs. Several algorithms have been proposed to solve this problem. \cite{navarro:guided} presents a survey of the problem and existing approaches to solving it at the time that work was written. Recently there have been some further work in the field, notably~\cite{maass:text} which also contains an updated overview of methods for solving the problem in addition to presenting one itself.

In this work we present an approach which solves this problem based on extending of suffix trees. The main advantage of this approach is that both the search and the index size are \emph{alphabet independent}, although the indexing time is not.

\subsection{Related Work}

Our work is related to efforts to index the text for approximate matching such as~\cite{maass:text,mosche}. The structure presented is superficially very similiar to the one presented in~\cite{inexact} as an \textit{inexact suffix tree} though that work is concerned with solving a different set of problems.

\subsection{Structure of the Paper}

In section~\ref{sec:structure} we present the structure and the way to use it for searching. In section~\ref{sec:build-1} we show the way to build a structure to handle one error only and analize this version of the algorithm. Then, we show, in section~\ref{sec:build-k} how to generalize this procedure for any number of errors and how the previous analisys also applies to any number of errors.

